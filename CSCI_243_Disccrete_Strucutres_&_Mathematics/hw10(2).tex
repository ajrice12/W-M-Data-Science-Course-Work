\documentclass[11pt]{article}

\usepackage{times,mathptm}
\usepackage{pifont}
\usepackage{exscale}
\usepackage{latexsym}
\usepackage{amsmath}
\usepackage{epsfig}

\textwidth 6.5in
\textheight 9in
\oddsidemargin -0.0in
\topmargin -0.0in

\parindent 0pt     % How much the first word of a paragraph is indented. 
\parskip 0pt       % How much extra space to leave between paragraphs.

\begin{document}

\begin{center}             % If you only centering 1 line use \centerline{}
\begin{LARGE}
{\bf CSci 243 Homework 10}
\end{LARGE}
\vskip 0.25cm      % vertical skip (0.25 cm)

Due: Monday Dec 4, end of day\\  % force new line
**My name**
\end{center}             % If you only centering 1 line use \centerline{}

Although most answers to the problems below are just one number,
 your responses should explain how this number was obtained.

\begin{enumerate}

\item A bowl contains 10 red balls and 10 blue balls.
You select balls at random without looking at them.
\begin{enumerate}
\item (4 points) How many balls must you select to be sure of having at least three balls of the same color?
\item (4 points) How many balls must you select to be sure of having at least three blue balls?
\end{enumerate}


\item (6 points) How many ordered pairs of integers $(a,b)$ are needed to guarantee that
there are two ordered pairs $(a_1,b_1)$ and $(a_2,b_2)$ such that $a_1\bmod 5=a_2\bmod 5$ and
$b_1\bmod 5=b_2\bmod 5$?

\item (6 points) How many numbers must be selected from the set $\{1,3,5,7,9,11,13,15\}$ to guarantee that at least one pair of these numbers add up to 16?


\item How many bit strings of length 12 contain
\begin{enumerate}
\item (3 points) exactly three 1s?
\item (3 points) at most three 1s?
\item (3 points) at least three 1s?
\item (3 points) an equal number of 0s and 1s?
\end{enumerate}

\item (8 points)
Assume $n$ distinct points on a circle and draw all line segments connecting each pair.
Find the number of intersections of all these line segments inside the circle.
Hint: You can find and analyze the formula recursively, using the formulas
$\sum_{i=1}^n i^3 = n^2(n+1)^2/4$, $\sum_{i=1}^n i^2 = n(n+1)(2n+1)/6$ and lots of algebra.\\
Alternative hint: observe that each intersection corrsponds to the diagonal of a rectangle. Use a combinatiorial method---no algebra needed.

\end{enumerate}

\end{document}
