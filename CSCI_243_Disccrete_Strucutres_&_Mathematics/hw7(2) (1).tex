\documentclass[11pt]{article}

\usepackage{multicol}
\usepackage{times,mathptm}
\usepackage{pifont}
\usepackage{exscale}
\usepackage{latexsym}
\usepackage{amsmath}
\usepackage{epsfig}

\textwidth 6.5in
\textheight 9in
\oddsidemargin -0.0in
\topmargin -0.0in

\parindent 0pt     % How much the first word of a paragraph is indented. 
\parskip 0pt       % How much extra space to leave between paragraphs.

\begin{document}

\begin{center}             % If you only centering 1 line use \centerline{}
\begin{LARGE}
{\bf CSci 243 Homework 7}
\end{LARGE}
\vskip 0.25cm      % vertical skip (0.25 cm)

Due: Wednesday, November 1, end of day \\  % force new line
**My name**
\end{center}

\begin{enumerate}

\item (10 points)
What is the largest problem size $n$ that we can solve in no more than
{\bf  one hour} using an algorithm that requires $f(n)$ operations,
where each operation takes $10^{-9}$ seconds (this is close to a today's computer),
with the following $f(n)$?
\begin{multicols}{2}
\begin{enumerate}
\item $\log_2 n$
\item $\log^4_2 n$
\item $3n$
\item $n\log_2 n$
\item $n\log^2_2 n$
\item $n^2$
\item $(3n)^3$
\item $2^n$
\item $n!$
\item $n^n$
\end{enumerate}
\end{multicols}


\item (10 points) 
Use pseudocode to describe an algorithm that determines whether a 
given function from a finite set to another finite set is one-to-one. Assume that the function $f: A\rightarrow B$ is given as a set of pairs $\{(a_i,f(a_i))\in A\times B, \forall a_i\in A\}$ and that the sets
$A =\{a_1,\ldots,a_m\}, B =\{b_1,\ldots,b_n\} $ are also given.

\item (5 points)
Describe an algorithm that produces the maximum, the minimum, and the mean, 
of a set of $n$ real numbers, passing through the numbers once. 


\item (10 points) 
Consider the following pseudocode that finds $x$ in a list of sorted
numbers by using ternary search. The algorithm is simlar to binary search, 
only it splits the current list into three parts (instead of two) and checks 
which part $x$ may be in. Thus at each step, the algorithm removes 2/3 of 
the items in the current list.
Find the complexity of the algorithm. Is it faster or slower asymptotically 
than binary search?
{\footnotesize
\begin{verbatim}
Algorithm TernarySearch(x int, a(1)...a(n) int, output: loc int)
i=1; 
j=n;
remaining = j-i+1;
while (remaining > 1)
   interval = floor(remaining/3);
   m1 = i+interval;
   m2 = i+2*interval;
   if (x <= a(m1)) 
      j=m1;
   else if (x<=a(m2))
      i=m1+1;
      j=m2;
   else 
      i=m2+1;
   endelsif
   remaining = j-i+1;
endwhile

if (remaining == 1 and x not equal a(i)) return loc=0;
else return loc = j;
\end{verbatim}
}

\item (5 points) Use bubble sort to sort 1, 3, 8, 2, 9, 4, showing the lists obtained 
at each outer step.



\end{enumerate}

\end{document}
